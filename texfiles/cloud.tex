\section{Load, Performance and Stress Cloud Testing}


Cloud computing alters the way of obtaining computing resources, managing and delivering software, technologies and solutions. Meanwhile, cloud computing also brings new issues, challenges, and needs in cloud-based application testing and evaluation. In the past decades, there were numerous published technical papers focusing on scalability analysis and performance evaluation. 

The presented research work found one systematic review on cloud software testing:


\begin{itemize}
\item 5W+1H pattern: A perspective of systematic mapping studies and a case study on cloud software testing \cite{Jia2015}.
\end{itemize}

The Figure \ref{fig:surveycloud}  shows a comparison between the  researches found in the presented research work. The x axis represents the kind of cloud used ( SaaS, PaaS  or IaaS )  and the y axis presents the type of workload strategy used by each research (Generativive or Descriptive Workload). The Figure also divides the researches by the type technique used in each paper.


\begin{figure}[!ht]
\centering
\includegraphics[width=0.5\textwidth]{./images/SurveyFiguraCloud3.png}
\caption{Distribution of the researches over kind of cloud}
\label{fig:surveycloud}
\end{figure}



\subsection{Papers found in the presented research work}

This subsection presents details about the papers found in the presented research work.

Gao et. al. proposes  a new formal graphic models and metrics to test SaaS performance and analyze system scalability in clouds. This paper presents a analytic models in a radar chart graphic format to evaluation for system performance and scalability of SaaS applications in a cloud.  The paper consider three types of system loads\cite{Gao2011}.

\begin{itemize}
\item The communication traffic load:  The amount of incoming and outgoing communication messages and transactions in a given time unit during system performance and evaluation.
\item The system user access load during:The number of concurrent users who access the system in a given time unit.
\item The system data load: the underlying
system data store access load, such as the number of data
store access, and data storage sizing.
\end{itemize}


Snellman et. al. propose the ASTORIA framework to identify problems using Record and Playback User Actions. The ASTORIA is a framework for automatic performance and scalability testing of Rich Internet Applications.\cite{Snellman2011}.

Tsai et. al. propose scalability metrics that can be used to test the scalability of SaaS applications.

Jayasinghe presents Expertus, a tool to automate large scale distributed experiment studies in IaaS clouds. The Expertus have the goal of addressing three challenges discussed in Section II.