\section{Planning and Conducting of Research}

In this section, the steps of planning and review of driving are presented in details. The first part of the study was a systematic review. 

A systematic review is a process of assessment of all available research related to a research question or subject of interest. The planning of the systematic review was carried out from the protocol defined by Biolchini \cite{Biolchini2005} \cite{Afzal2009}. Systematic literature review (SLR) is a approach to conducting survey on a research topic. It aims at producing an "engineering" approach with a well-defined methodology so that different investigators can produce survey results effectively and reliably \cite{Kitchenham2007}.


Planning is the starting point for the review, whose main points are the definition of one or more research questions. The activities to conduct the research  includes formulate Research Questions,  The Paper idenfification process (selection of sources to search, search strategies and the use of keywords). 

A  key activity in the planning  of a typical SLR project  is to formulate a set of Research Questions (RQs) before identifying, reading, and analyzing articles of the topic \cite{Jamshidi2013}. 


\subsection{Research Questions}

The work aims to answer two research questions:

First Research Question : How to define a suitable workload for Load Testing to generate realistic loads models (Realistic Workloads)?

Second  Research Question : How to set a suitable workload for Load Testing to fault-inducing loads ?

Third Research Question: In which fora is research on load, performance and stress testing published? 

Fourth Research Question: Which topics for load, performance and stress testing  have been investigated and to what extent?

Fifth Research Question: What types of research are represented and to what extent? 

\subsection{Search strategy}


In this section,we present the search strategy of our paper. 
For each question, keywords were chosen and used in a search strategy. The search strategy for the selection of studies was carried out through search in repositories (ACM, Springer, IEEE, Google Scholar, Science Direct, Mendeley ) , language ( Portuguese and English) and the keywords defined. Using the results, new keywords have been included, feeding back the process. The research strategy included these two practices:

%(Figure \ref{fig:figuraselecao})

\begin{enumerate}
\item Identification of other words and synonyms for terms used in the research questions. This practice is used for minimize the effect of differences in terminologies;
\item The keywords and their possible combinations and synonyms were submitted in the selected repositories search engines; 
\item  Among the results, were excluded studies not related to load, performance and stress tests.
\end{enumerate}

%\begin{figure}[!ht]
%\centering
%\includegraphics[width=0.50\textwidth]{./images/surveybusca2.png}
%\caption{Search Strategy}
%\label{fig:figuraselecao}
%\end{figure}



We used the following search terms:

\begin{itemize}

\item Stress Testing: Search-based Testing, Genetic Algorithms, Stress Testing, Test Tools, Test Automation, Empirical Analysis, Denial of Service, Ramp-Up time, Think Timer,  Response Time, Bandwidth Throttle, Dynamic Stress Testing, Evolutionary, Heuristic, Search-Based, Metaheuristic. optimization, genetic algorithms, genetic programming.
\item Performance Testing: Performance Testing, Web-based Systems, Software Testing, Model-Based Testing, Software Product Line, Regression Testing, Test Failure Prediction, Genetic Metric Selection.
\item Load Testing: Markov chain,  Automatic Test Case Generation Algorithms, Domain-based reliability measure, Fault detection, Load Test suites, load testing, Reliability, Resource allocation mechanisms, Software testing, System degradation.


<<<<<<< HEAD
\end{itemize}


\subsection{Study selection procedure}

The selected studies were filtered by one researcher that used the following inclusion or exclusion criteria:

\begin{itemize}
\item Include: The researcher is sure that the paper is in scope and that it was properly validated using empirical methods.
\item Exclude: The researcher is sure that the paper is out of scope or that the validation was insufficient.
\item Uncertain: The researcher is not sure whether the paper fulfills either the inclusion or exclusion criteria above.
\end{itemize}


\subsection{Classification scheme}

The classification used by the study  is a structure of empirical studies on load,. The scheme consists of six facets, namely quantification approach, abstraction, context, evaluation, research method, and measurement purpose (see Fig. 1). Classification schemes/taxonomies are rated based on a set of quality attributes. A good taxonomy/classification is:
1. Orthogonality: There are clear boundaries between categories, which makes it easy to classify.
2. Defined based on existing literature: The taxonomy/classification is created based on an exhaustive analysis of existing literature in the field.
3. Based on the terminology used in literature: The taxonomy uses terms that are used in existing literature.
4. Complete: No categories are missing, so that existing articles can be classified.
5. Accepted: The community accepts and knows the classification/ taxonomy.


In order to classify each paper selected in accordance with the characteristics of the survey, 



\subsection{Data Extraction and Mapping of Studies}

The relevant articles are sorted into classification scheme, i.e., the actual data extraction takes place. As shown in Figure  the classification scheme evolves while doing the data extraction, like adding new categories or merging and splitting existing categories. 


