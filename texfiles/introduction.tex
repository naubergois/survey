\section{Introduction}

Many systems must support concurrent
access by hundreds or thousands of users. The failure to scale users results in catastrophic failures and unfavorable media coverage\cite{Jiang2010}. To assure the quality of these systems, performance, stress and load testing is a required testing procedure\cite{Jiang2009}. 

The explosive growth of the Internet has contributed to  increase the need for applications to perform at warp speed. Performance problems have a bad habit of turning up late in the application life cycle, and the later you discover them, the greater the cost to fix them \cite{Molyneaux2009}.
% * <naubergois@gmail.com> 2015-09-16T23:58:29.774Z:
%
%  Alterar frase
%
% ^ <naubergois@gmail.com> 2015-09-16T23:58:46.491Z.
% * <naubergois@gmail.com> 2015-09-16T23:58:52.202Z:
%
% 
%
The use of load testing is an increasingly common practice due to the increasing number of users. In this scenario, the inadequate treatment of a workload generated by concurrent or simultaneously access, generated by system users, can result in highly critical failures and corrosion of the company's image in their customers' view \cite{Draheim2006b} \cite{Jiang2010}. 
% * <naubergois@gmail.com> 2015-09-16T23:58:52.451Z:
%
%  Frase coloquial
%
% ^ <naubergois@gmail.com> 2015-09-16T23:59:03.275Z.

An analysis of sources of failures in the United States Public Switched Telephone Network (PSTN) reported that only 6\% of the outages were overloads. These interruptions have produced  44\% of the PSTN's service's downtime \cite{Kuhn1997}. 

The Load Testing determines the responsiveness, throughput, reliability or scalability of a system under a given workload. The quality of the results of system's load tests is closely linked to the implementation of the workload strategy. The performance of many applications depends on the load applied under different conditions. In some cases, performance degradation and failures arise only in stress conditions \cite{Garousi2010} \cite{Jiang2010}.


Different parts of an application should be tested on various parameters and stress conditions \cite{Babbar2011}. The correct application of a load test should cover most part of application under ordinary conditions (Load or Performance Test) or above the expected load conditions(Stress Test) \cite{Draheim2006b} \cite{Luiz2011} \cite{Fe2004}.

Evolutionary testing is seen as a promising approach for verifying timing constraints \cite{Afzal2009a}. In evolutionary testing the search for the longest execution time is considered a discontinuous, nonlinear optimization problem, with the input domain of the test object as search space, sets of test data as decision variables, and execution times as objective values \cite{Stations}. The main objective of load, performance and stress evolutionary testing is to find test scenarios which produce execution times violating the specified timing constraints \cite{Sullivan}. 

% * <naubergois@gmail.com> 2015-09-17T00:49:26.764Z:
%
%  Rever paragrafo abaixo
%
%
The purpose of this paper is propose the use of a approach using  hybrid metaheuristc   with  Genetic Algorithms, Simulated Annealing and Tabu Search Algorithms  in load, performance and stress evolutionary tests.

The remainder of the paper is organized as follows. Section 2 presents a brief introduction in load, performance and stress tests. Section 3 presents concepts about workload modeling. Section 4 presents a brief introduction about evolutionary test definitions, techniques and state of art. Section 5 presents the IAdapter tool. The Section 6 shows the results of two experiment applied with IAdapter. Conclusions and further work are presented in Section 7.